\section{Literature Review}
Creating lower frequency waves from higher frequency waves by non-linear interaction began in the field of sonar through development of underwater sonar techniques as far back as the 1960's \cite{westervelt_1963}. These developments produced a formal mathematical basis for the effect of directive ultrasonic transducers and were referred to as parametric arrays.\\
While these sonar systems were developed for use underwater, a publication in 1975 \cite{bennett_blackstock_1975} revealed that the nonlinear effects observed underwater could occur in air as well.\\
By 1983, companies such as Ricoh \cite{yoneyama_fujimoto_kawamo_sasabe_1983} developed directional loudspeaker systems; however, found them quite costly to consider as a viable product for exhibits and museum installations. Their implementations were rudimentary with basic equalisation of the audio signal and double side band amplitude modulation. They identified a proportional relationship between the modulation depth ($m)$ and the sound pressure of the signal; however, found that the distortion is proportional to $m^2$.\\
The following year in 1984, Kamakura [ref kamakura] proved that the distortion can be reduced by preprocessing this signal appropriately. Pompei [ref pompei] implemented the square root amplitude modulation (SRAM) technique proposed by Kite [ref kite] which overcame the squaring of the envelope signal first identified by Berktay's approximation. This reduced the harmonic distortion compared to Double Side Band Amplitude Modulation (DSBAM) used before.\\
Kite [ref kite] noted that the distortion can be totally removed if the harmonics of the modulationg signal created by the square root process in SRAM are reproduced by the ultrasonic transducers themselves. However, this would require an infinite bandwidth in the ideal case or atleast more than 10kHz of bandwidth which aren't feasible.\\
The past innovations in the field neglected to consider the bandwidth of the transducers themselves. In 2009, [Tan, Ji \& Gan ref here (On preprocessing techniques for bandlimited parametric loudspeakers)] tackled this problem by considering a different modulation approach using Modified Amplitude Modulation (MAM). This modulation scheme involved a form of quadrature amplitude modulation which took into account the 3dB bandwidth of the transducers themselves. This further lowered the distortion present in the demodulated signal.

Recent studies suggest that this harmonic distortion is related to the frequency response of the ultrasonic transducers not being considered in Berktay's equation; however, a paper from 2018 \cite{farias_abdulla_2018} suggests this can be overcome by choosing a suitable modulation technique and applying a filter which describes the transducer array's frequency response.\\
