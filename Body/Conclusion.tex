\section{Discussion \& Conclusion}
This section discusses the results from the tests and the project as a whole. The results for the entire directional audio system are then summarised in the conclusion while providing suggestions for future avenues of work for the project.
\subsection{Discussion}
The results from the distortion and directivity tests provided real world data on the low cost implementation of a directional audio system. Due to the real world nature of these results, some noise is present which was minimised as best as possible.\\
\paragraph{Distortion result discussion}
The distortion results for the traditional loudspeaker came out as expected. A large 2.5 kHz tone was present with relatively small magnitude low frequency noise. Unexpectedly, the traditional loudspeaker produced a very low magnitude 1st harmonic at 5 kHz. This 1st harmonic's presence could likely be an artefact in the julia code used to produce this test tone as the output may have clipped what the audio jack is capable of. While the 1st harmonic's presence is visible in the spectrum of the recorded data, it is not discernible to human hearing and should not be considered a significant form of distortion due to its low relative magnitude compared to the 2.5 kHz fundamental and as such is a good control for the tests.\\
The ultrasonic directional speaker distortion results provided insight into what the pre-processing does to the magnitude of the harmonics. From Kite's results \cite{kite_post_hamilton_1998}, one expects the square-root AM (SRAM) technique to result in lowered harmonic distortion compared to the pure AM case. The results show that the pure AM output without pre-processing has seemingly asymmetric distribution of harmonic magnitudes while the SRAM pre-processed results demonstrate a more natural and symmetrical distribution of harmonics. The high magnitude of the 1st harmonic is an unexpected result but is shown to perhaps be an issue with the interaction of the ultrasonic beam and the microphone since the 2.5 kHz tone is the strongest magnitude when the beam is slightly off centre from the microphone. Close inspection of these spectrums reveals small magnitude side tones of each of the major tones which appear to be similar to what was shown in the signal simulations for the square-rooted signal featuring a large exponential decay of these side tones. The higher frequency harmonics are undesirable as they create distortion, however; the distortion is reduced for the SRAM pre-processed result since its harmonics all appear to drop off at a near exponential rate. The presence of the large 5 kHz tone could have to do with the beam reflecting off of the wall behind the microphone since the off beam result shows a stronger presence of the 2.5 kHz tone. This 5 kHz presence could also be the result of an imperfect squaring by the environment due to the small distance between the transducers and the microphone; resulting in a difference frequency slightly off the expected baseband frequency seen in simulations. One final factor that could be the issue is the use of large carrier amplitude modulation. Since the carrier is larger than the side lobes where the signal lies, the difference frequency that the environment creates during its demodulation could result in this carrier adding energy to a particular frequency in its side bands due to harmonics.\\
\paragraph{Directivity result discussion}
The directivity results for the traditional loudspeaker showed a relatively consistent signal level through its sweep. There were some notable troughs in the filtered cases which are likely the result of the environment reflecting and absorbing varying amounts of sound pressure. For all the filtered cases, no main peak in signal level coincided with the beam crossing the centre of the microphone. This demonstrates the relatively low directivity of conventional loudspeakers and aside from some inconsistencies in signal level due to reflections in the testing environment, can be considered a good control representation of a non-directional loudspeaker.\\
The results for directionality testing of the directional audio system demonstrated the real world representation of what the system sounds like to human hearing relatively well. When the beam is experienced, a clear difference in sound is heard depending on which ear is in the beam. During testing only one microphone was used and the waveform it recorded represents what is heard quite well. Before the directional audio beam reaches the microphone, subtle magnitudes of the test tone can be heard in the environment as shown by the 2.5 kHz filtered result. As the beam approaches the microphone, a sudden increase in volume occurs, thus demonstrating significantly more directionality than the loudspeaker. A similar gradient in drop off of signal loudness is experienced as the microphone exits the beam until the signal level is returned to the level before the beam reached the microphone. Further filtering of the 1st and 2nd harmonic revealed that these tones are more directional than the original test tone of 2.5 kHz. Since the beam angle is directly proportional to wavelength, thus; inversely proportional to frequency as per equation \ref{eqn:beamwidthEqn}, this result makes some sense as the beam angle is reduced as the frequency increases and should yield a tighter beam width. Investigating the magnitudes of the filtered results reveals that both the fundamental and the 1st harmonic are approximately equally audible which is undesirable but the fundamental tone is still distinguishable. The 2nd harmonic is hardly audible but still present which is more desirable since it makes it less distinguishable.

\paragraph{Directional audio system development discussion}
The development of the directional audio system as a whole explored many facets of engineering and resulted in a functional prototype producing a functional directional audio beam. The resultant output of the system may not be ideal as distortion is prevalent in the system, however; the prototype system manages to prove that a low cost directional audio system is possible. During implementation it was discovered that the integration of the signal only served to further distort the signal and resulted in the application only square-rooting the signal during pre-processing. This coincides with prior works discussed during the literature review where no integration was performed on the signal despite the presence of the double derivative in Berktay's far-field solution \cite{berktay_1965} shown in equation \ref{eqn:berktayRelationship}. \\%This irrelevance of integration could be due to the inability for human hearing to detect phase changes, thus implying humans have little sensitivity to time variance in audio signals.\\
The designs for the system focused quite heavily on the ultrasonic transducer array since there has been little discussion of its design in the prior works found during the literature review. The design process provided helpful insight into selecting an appropriately packed array given the constraints of the project and may be further iterated on in a future implementation of this project. A intriguing investigation into modular transducer array PCBs was explored and discovered little benefit compared to a monolithic design since the output performance of the system was prioritised. A future implementation of this project may revisit this modular approach as scaling up the existing array design to hold more transducers would require new packing simulations and overall redesign of the array.\\
The auditory experience the system created was unlike any listening experience encountered before. The location of the sound source was clearly evident when transitioning in and out of the radiated beam as far as 4.5m away. As the beam passed from one ear to the next, the audio experienced sharp changes in intensity for each ear while giving the apparent experience that the sound was inside ones own head when the beam was centred on ones face. Additionally, the beam appears to reflect quite well off of non-absorbent surfaces. The resultant reflection maintained its directional effect (while reducing in intensity) and gave the illusion that the reflecting surface was now the source of the sound.
\subsection{Conclusion}
To conclude, the project successfully achieved the project goal of producing a simple, low-cost prototype of a directional audio system purely to demonstrate the directional audio effect. Each developmental milestone was achieved and produced an output for the next developmental milestone. All these goals were achieved well within budget featuring a cost of under ZAR 1000 mostly due to the use of software for pre-processing of audio. The resultant directional audio system successfully achieves a directional ultrasonic beam which self-demodulates in the environment and creates a directional audio beam as demonstrated by the directivity results. The audio beam suffers from 1st and 2nd harmonic distortion which is undesirable, but could not be corrected within the scope of this project due to time and cost limitations. Furthermore, the distortion tests reveal that the 1st harmonic has a larger presence than the fundamental frequency of the test tone when the beam strikes the microphone head on, however; the fundamental frequency is stronger than the harmonics when the beam is off centre of the microphone. This was suspected to be an issue relating to the large carrier AM used by the mixer. The apparent directionality of the system to human hearing subjectively appears highly directional due to the stereo hearing capability of human hearing. The source of the sound is easily identifiable when in the beam and less so when outside of it. This beam was easily heard at distances up to 4.5m away with little widening of the beam. The audio beam also appeared to be able to reflect off non-absorbent surfaces and maintaining its directionality, resulting in the surface appearing as the sound source. The project as a whole could be further improved on as it was heavily restrained in scope due to the Covid-19 pandemic of 2020 reducing access to many developmental resources offered by UCT. This restrained scope results in the project having many avenues for improvement which are discussed in the next subsection.

\subsection{Future improvements}
The directional audio system produced an effective proof of concept demonstrating the directional audio effect. Unfortunately the resultant audio suffered from significant 1st and 2nd harmonic distortion which is undesirable in a directional loudspeaker. A future implementation of this system could explore alternative modulation techniques and identify which technique results in a lowered cost implementation with a higher quality audio beam.\\
%speak about characterising transducer freq and tranfer func and implementing it in sim
Another facet of research includes the characterisation of the transducers for use in simulation. The frequency response and transfer function of the transducer could be measured and imported into simulations, allowing for a more realistic signal simulation of the entire signal chain. This could help in discovering what modulation and pre-processing techniques work best with the limited bandwidth of the transducers in use.\\
The transducer array designed in this project produced a relatively strong directivity compared to a traditional speaker but struggled to produce significant sound pressure levels. A future improvement could investigate alternative transducers or even larger scale transducer arrays and how they effect the resultant output power of the system. This improvement may want to investigate the use of alternatively shaped modular PCBs for the array as the solution in this project is not easily scalable.\\
%using a more directional microphone
Finally, due to challenges external to the project; resources for testing were limited. Ideally the incident angle of the beam relative to the microphone would have produced far more intuitive test results than what was shown in this report. A future implementation may want to consider placing the array on a platform with remotely controllable rotation increments. Then capturing recordings at set angles relative to the microphone and processing this into a beam pattern plot for a more conventional directivity result.